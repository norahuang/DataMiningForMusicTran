%
% This file presents the 'authoryear-ibid' style
%
\documentclass[a4paper]{article}
\usepackage[T1]{fontenc}
\usepackage[utf8]{inputenc}
\usepackage[american]{babel}
\usepackage{csquotes}
\usepackage[style=authoryear-ibid,backend=biber]{biblatex}
\usepackage{hyperref}
\addbibresource{biblatex-examples.bib}
% Some generic settings:
\newcommand{\cmd}[1]{\texttt{\textbackslash #1}}
\setlength{\parindent}{0pt}
\newenvironment{bibsample}
  {\trivlist\samepage
   \setlength{\itemsep}{0pt}}
  {\endtrivlist}
\begin{document}

\section*{The \texttt{authoryear-ibid} style}

This citation style is a variant of the \texttt{authoryear} style.
Immediately repeated citations are replaced by the abbreviation
`ibidem' unless the citation is the first one on the current page or
double page spread (depending on the setting of the
\texttt{pagetracker} package option). This style is intended for
citations given in footnotes.

\subsection*{Additional package options}

\subsubsection*{The \texttt{ibidpage} option}

The scholarly abbreviation \emph{ibidem} is sometimes taken to mean
both `same author~+ same title' and `same author~+ same title~+ same
page' in traditional citation schemes. By default, this is not the
case with this style because it may lead to ambiguous citations. If
you you prefer the wider interpretation of \emph{ibidem}, set the
package option \texttt{ibidpage=true} or simply \texttt{ibidpage} in
the preamble. The default setting is \texttt{ibidpage=false}.

\subsubsection*{The \texttt{dashed} option}

By default, this style replaces recurrent authors/editors in the
bibliography by a dash so that items by the same author or editor
are visually grouped. This feature is controlled by the package
option \texttt{dashed}. Setting \texttt{dashed=false} in the
preamble will disable this feature. The default setting is
\texttt{dashed=true}.

\subsubsection*{The \texttt{mergedate} option}

Since this style prints the date label after the author/editor in the
bibliography, there are effectively two dates in the bibliography:
the full date specification (e.g., \enquote{2001}, \enquote{June
2006}, \enquote{5th~Jan. 2008}) and the date label (e.g.,
\enquote{2006a}), as found in citations. The \texttt{mergedate}
option controls whether or not date specifications are merged with
the date label. This option is best explained by example. Note that
it only affects the bibliography. Citations use the date label only:

\begin{bibsample}
\item Doe 2000
\item Doe 2003a
\item Doe 2003b
\item Doe 2006a
\item Doe 2006b
\end{bibsample}

\texttt{mergedate=false} strictly separates the date specification
from the date label. The year will always be printed twice:

\begin{bibsample}
\item Doe, John (2000). \emph{Book~1}. Location: Publisher, 2000.
\item Doe, John (2003a). \emph{Book~2}. Location: Publisher, 2003.
\item Doe, John (2003b). \emph{Book~3}. Location: Publisher, 2003.
\item Doe, John (2006a). \enquote{Article~1}. In: \emph{Monthly Journal} 25.6
(June~2006), pp.~70--85.
\item Doe, John (2006b). \enquote{Article~2}. In: \emph{Quarterly Journal} 14.3
(Fall~2006), pp.~5--25.
\end{bibsample}

\texttt{mergedate=minimum} merges the dates whenever the full date
and the date label are exactly the same string. If the date is a bare
year number and there is no \texttt{extrayear} field, the date
specification will be omitted:

\begin{bibsample}
\item Doe, John (2000). \emph{Book~1}. Location: Publisher.
\item Doe, John (2003a). \emph{Book~2}. Location: Publisher, 2003.
\item Doe, John (2003b). \emph{Book~3}. Location: Publisher, 2003.
\item Doe, John (2006a). \enquote{Article~1}. In: \emph{Monthly Journal} 25.6
(June~2006), pp.~70--85.
\item Doe, John (2006b). \enquote{Article~2}. In: \emph{Quarterly Journal} 14.3
(Fall~2006), pp.~5--25.
\end{bibsample}

\texttt{mergedate=basic} is similar in concept but more economical.
It will always omit the date specification if the date is a bare year
number:

\begin{bibsample}
\item Doe, John (2000). \emph{Book~1}. Location: Publisher.
\item Doe, John (2003a). \emph{Book~2}. Location: Publisher.
\item Doe, John (2003b). \emph{Book~3}. Location: Publisher.
\item Doe, John (2006a). \enquote{Article~1}. In: \emph{Monthly Journal} 25.6
(June~2006), pp.~70--85.
\item Doe, John (2006b). \enquote{Article~2}. In: \emph{Quarterly Journal} 14.3
(Fall~2006), pp.~5--25.
\end{bibsample}

\texttt{mergedate=compact} merges all date specifications with the
date labels. It will still treat the \texttt{issue} field specially:

\begin{bibsample}
\item Doe, John (2000). \emph{Book~1}. Location: Publisher.
\item Doe, John (2003a). \emph{Book~2}. Location: Publisher.
\item Doe, John (2003b). \emph{Book~3}. Location: Publisher.
\item Doe, John (June 2006a). \enquote{Article~1}. In: \emph{Monthly Journal} 25.6, pp.~70--85.
\item Doe, John (2006b). \enquote{Article~2}. In: \emph{Quarterly Journal} 14.3
(Fall), pp.~5--25.
\end{bibsample}

\texttt{mergedate=maximum} strives for maximum compactness. Even the
\texttt{issue} field is merged with the date label:

\begin{bibsample}
\item Doe, John (2000). \emph{Book~1}. Location: Publisher.
\item Doe, John (2003a). \emph{Book~2}. Location: Publisher.
\item Doe, John (2003b). \emph{Book~3}. Location: Publisher.
\item Doe, John (June 2006a). \enquote{Article~1}. In: \emph{Monthly Journal} 25.6, pp.~70--85.
\item Doe, John (Fall 2006b). \enquote{Article~2}. In: \emph{Quarterly Journal} 14.3, pp.~5--25.
\end{bibsample}

\texttt{mergedate=true} is an alias for \texttt{mergedate=compact}.
This is the default setting.

\subsection*{Hints}

If you want terms such as \emph{ibidem} to be printed in italics,
redefine \cmd{mkibid} as follows:

\begin{verbatim}
\renewcommand*{\mkibid}{\emph}
\end{verbatim}

\subsection*{\cmd{footcite} examples}

This is just filler text.\footcite{companion}
% Immediately repeated citations are replaced by the
% abbreviation 'ibidem'...
This is just filler text.\footcite{companion}
\clearpage
% ... unless the citation is the first one on the current page
% or double page spread (depending on the setting of the
% 'pagetracker' package option).
This is just filler text.\footcite[55]{companion}
This is just filler text.\footcite[55]{companion}

\clearpage
\printbibliography

\end{document}
